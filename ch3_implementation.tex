[[section: Andriod Application]]
	[[subsection: UX/UI]]
		การออกแบบ UI ของ Andriod Application เราได้จำลองสวนควบคุมรถที่สำคัญมาเป็นส่วนประกอบต่างๆบนหน้าจออุปกรณ์สมาร์ตโฟน ซึ่งได้แก่ เบรคและคันเร่ง พวงมาลัย และเกียร์ และเพิ่มส่วนประกอบอื่นๆที่สำคัญเข้าไปคือ ส่วนของการเปลี่ยนโหมดควบคุม และแสดงผลภาพเคลื่อนไหวจากกล้อง
		% แทรกโคด

	[[subsection: Low-level Socket Programing]]
		การใช้อุปกรณ์สมาร์ตโฟนควบคุมรถนั้น จำเป็นต้องให้อุปกรณ์และรถเชื่อมต่อและรับ-ส่งข้อมูลกันได้ ซึ่งในส่วนนี้บน Android Application จึงต้องใช้ Socket Programming โดยเลือก TCP เป็น Protocal ที่ใช้ในการสื่อสาร 
		เนื่องจากระบบต้องการความถูกต้องและแม่นยำของข้อมูล เพื่อให้แน่ใจว่าข้อมูลที่ได้รับ และส่งกลับมามีความถูกต้องแน่นอน
		% แทรกโคด

	[[subsection: MJPEG Sreaming]]
		การรับภาพมาจากรถผ่าน HTTP โปรแกรมต้องทำการอ่านข้อมูลจาก Streaming จากนั้นนำมาตรวจสอบว่าข้อมูลที่ได้คือภาพหรือไม่ หากเป็นภาพจะทำการสร้างภาพบน SurfaceView ซึ่งเป็น Class ที่ใช้แสดงผลของ Android SDK
		% แทรกโคด

[[section: รถจำลอง]]
	[[subsection: ส่วนของ Software]]
		ใช้ภาษา Python ในการพัฒนาโดยจะเรียกซอฟแวร์ทั้งหมดว่า Server และแยกเป็น Module ซึ่งโมดูลที่สำคัญประกอบไปด้วย

		[[subsubsection: Socket Controller ]]
			เป็นส่วนที่ใช้ในการสื่อการกับ Client ทั้งหมด ไม่ว่าจะเป็นการรับ หรือส่งข้อมูล ในส่วนนี้จะทำการรับข้อมูลและนำไปประมวลผลต่อไป
			% แทรกโคด

		[[subsubsection: System Controller]]
			เป็นตัวกลางระหว่าง Socket Controller กับ Hardware Controller 
			% แทรกโคด

		[[subsubsection: Hardware Controller]]
			เป็นส่วนสั่งการอุปกรณ์ต่างๆในตัวรถเช่น มอเตอร์ขับเคลื่อน เซอโวล้อ เซอโวกล้อง
			% แทรกโคด
			

	[[subsection: ส่วนของ Hardware]]

		[[subsection: DC Motor]]
			% Content
			% แทรกภาพ

		[[subsection: Wheels Servo]]
			% Content
			% แทรกภาพ
		
		[[subsection: Camera Servo]]
			% Content
			% แทรกภาพ

		[[subsection: H-bridge Circuit]]
			% Content
			% แทรกภาพ
	
[[section: ชุดขับจำลอง]]
	[[subsection: ส่วนของ Software]]
		

	[[subsection: ส่วนของ Hardware]]

