[[section: อุปกรณ์ที่ใช้ในการพัฒนา ]]
	[[subsection: อุปกรณ์ด้านซอฟต์แวร์ ]]
		[[list]]
		# Sublime text 3
		# Android Studio
		# Putty
		# WinSCP
		# XShell
		[[end]]

	[[ subsection: อุปกรณ์ด้านฮาร์ดแวร์ ]]
		[[list]]
		
		# บอร์ด Linkitsmart 7688
		# DC Motor ขนาด 38 มม.
		# เซอร์โวมอเตอร์ 2 ตัว
		# กล้อง(รองรับ UVC)
		# แบตเตอรี่ 1500 mAh 
		[[end]]

%%%%%%%%%%%%%%%%%%%%  END 1  %%%%%%%%%%%%%%%

[[section: รายละเอียดการพัฒนา]]	

	[[ subsection: การติดต่อสื่อสาร ]]
		การพัฒนาให้ตัวควบคุมสามารถควบคุมรถได้นั้นจำเป็นต้องมีการสร้างหรือกำหนดลักษณะเฉพาะของการสื่อสารขึ้ันมา ซึ่งนอกเหนือจากเทคโนโลยีไวไฟที่ใช้เชื่อมต่อเซิฟเวอร์กับไคลแอนท์เข้าด้วยกันแล้ว การแลกเปลี่ยนข้อมูลจริงๆนั้นจำเป็นต้องใช้ Socket Programing เพื่อสื่อสารกัน โดยทำการเลือกใช้ TCP เป็นโพรโตคอลหลักในการสื่อสาร เพราะการส่งข้อมูลจำเป็นต้องมีความแม่นยำ และรับประกันว่าการส่งข้อมูลแต่ละครั้งผู้รับจะได้รับข้อมูลแน่นอน

		[[image(tcp_flow):tcp_flow.png|width=0.5|caption=ภาพการทำงานของ TCP]]

		จาก [[ ref(tcp_flow) ]] แสดงให้เห็นว่าการส่งข้อมูลผ่าน TCP นั้นมีการตรวจสอบทุกครั้งก่อนส่งข้อมูล ทำให้ข้อมูลที่ส่งไปถึงผู้รับแน่นอน

		[[subsubsection: รูปแบบของข้อมูลที่ใช้ในการสื่อสาร ]]
			การส่งข้อมูลระหว่างไคลแอนท์ และเซิฟเวอร์มีรูปแบบเป็นข้อความเพื่อระบุคำสั่งให้ผู้รับนำไปประมวล ลักษณะข้อความที่ใช้ในระบบมี 2 แบบ ดังนี้
			[[list ]]
			# non-command
				คือการส่งข้อความที่มีลักษณะยาวติดต่อกันไม่มีการเว้นวรรคประกอบด้วยหัวและค่าที่ต้องการ
				[[table[| l | l | l |](Control Data): ตารางการส่งข้อมูลแบบ non-command ]]
				\hline
				$ส่วนหัว$ & $ข้อมูล$ & $ คำอธิบาย $\\
				\hline
				$a$ & $ 0 - 100$ & $ คำสั่งคันเร่งมีค่าเป็นจำนวนเต็ม 0 - 100 $\\
				$b$ & $ 0 - 100$ & $ คำสั่งเบรคมีค่าเป็นจำนวนเต็ม 0 - 100 $\\
				$t$ & $ 0 - 180$ & $ คำสั่งเลี้ยวมีค่าเป็นองศาจำนวนเต็ม 0 - 180 $\\
				\hline
			[[end]]

			# command
				มีลักษณะคล้ายกับคำสั่งในระบบปฏิบัติการ มีการแยกคำสั่งกับค่าที่ระบุด้วยการเว้นวรรค เช่น -cm SIM เป็นการส่งข้อความเพื่อขอเปลี่ยนโหมดควบคุม
				[[table[| l | l | l |](Command Data): ตารางการส่งข้อมูลแบบ command]]
				\hline
				$ คำสั่ง $ & $ อาร์กิวเมนต์ $ & $ คำอธิบาย $\\
				\hline
				$ -a $ & $ “PHONE”, “SIMULATOR_SET” $ & $ คำสั่งการยืนยันตัวตนของอุปกรณ์ที่เชื่อมต่อกับ Server $\\
				$ -cm $ & $ “PH”, “SIM”  $ & $ คำสั่งเปลี่ยนโหมดการควบคุม ตัวอย่าง “-cm PHONE”  $\\
				$ -cc $ & $ 0 - 1 $ & $ คำสั่งควบคุมกล้องมีหน่วยเป็น omega มีค่าเป็นเลขทศนิยม 0 - 1 ตัวอย่าง “-cc 0.333” $\\
				$ -ccp $ & $ 0 - 180 $ & $ คำสั่งควบคุมกล้องมีหน่วยเป็นองศา มีค่าเป็นเลขจำนวนเต็ม 0 - 180 ตัวอย่าง “-cc 0.333” $\\
				\hline
				[[end]]

			[[end]]																

			[[subsubsection: รูปแบบการสื่อสาร ]]

			ในการสื่อสารและแลกเปลี่ยนข้อมูล บางข้อมูลจะต้องมีการโต้ตอบกัน และบางข้อมูลไม่จำเป็นต้องมีการโต้ตอบ กล่าวคือการส่งข้อมูลบางอย่างที่ผู้รับและผู้ส่งจะส่งข้อมูลหากันจากนั้นนำข้อมูลที่ได้ไปประมวลผล และตอบกลับไป 
					
				[[list]]
					# การส่งข้อมูลแบบทางเดียว ใช้กับการส่งคำสั่งควบคุมรถซึ่ง ไคลแอนท์เป็นผู็ส่งคำสั่งไปให้เซิฟเวอร์  แต่เซิฟเวอร์ไม่จำเป็นต้องส่งข้อความตอบกลับมา
					แทรกภาพ		
					# การส่งข้อมูลสองทางแบบสมบูรณ์ เป็นการส่งข้อมูลกันแบบที่ผู้ส่งจะได้รับข้อมูลตอบจากผู้รับเช่น การส่งข้อมูลเพื่อเปลี่ยนโหมดควบคุม เมื่อทำการเปลี่ยนโหมดสำเร็จเซฺฟเวอร์จะส่งข้อความกลับเพื่อให้อุปกรณ์โทรศัพท์ทำการอัพ UI 

					[[image(data_flow_control):data_flow_control.jpg|width=0.5|caption=รูปแบบการทำงานของ Data Flow Control]]

			 		จาก [[ ref(data_flow_control) ]] จะเห็นได้ว่าข้อมูลที่ผู้ส่งส่งไปนั้นจะผ่านการตรวจสอบที่ผู้รับก่อนที่จะมีการโต้ตอบกลับไปจึงเป็นการส่งข้อมูลแบบสองทางสมบูรณ์  ดังนั้นหากข้อมูลมีการส่งไปแล้วผู้รับไม่มีการตอบกลับ ก็จะถือว่าข้อมูลดังกล่าวถูกปฏิเสธจึงไม่มีการส่งข้อมูลกลับไป ซึ่งเป็นการส่งข้อมูลแบบสองทางไม่สมบูรณ์ 		
					#การส่งข้อมูลสองทางแบบไม่สมบูรณ์ เป็นการส่งข้อมูลที่ผู้รับไม่ได้ตอบกลับไปยังผู้ส่งเพราะเงื่อนไขของคำสั่งไม่ถูกต้อง เช่นทำการเปลี่ยนเกียร์ขณะใช้ความเร็ว เซิฟเวอร์จะไม่ตอบกลับไปยังไคลแอนท์		
				[[end]]

			[[image(change_mode_data_flow):change_mode_data_flow.png|width=0.5|caption=ตัวอย่างการสื่อสารเพื่อนเปลี่ยนโหมดควบควบเป็น Simulator ]]

%%%%%%%%%%%%%%%%%%%%  END 2  %%%%%%%%%%%%%%%

	[[subsection: การพัฒนารถจำลอง]]    
	    เป็นการจำลองรถจากอัตราส่วนจริง  _อัตราส่วน_ ภายในประกอบไปด้วย บอร์ด LinkIt Smart 7688 Duo ซึ่งโปรแกรมด้วยภาษาไพธอน เป็นตัวควบอุปกรณ์ต่างๆได้แก่ มอเตอร์ และเซอโว โดยจะรับส่งข้อมูลด้วยเทคโนโลยีไวไฟ

		[[subsubsection: ส่วนของ Software]]
			ใช้ภาษาไพธอน ในการพัฒนาโดยจะเรียกซอฟแวร์ทั้งหมดว่า Server และแยกเป็น Module ซึ่งโมดูลที่สำคัญประกอบไปด้วย

	 	   	[[subsubsection: Socket Controller ]]
	            เป็นส่วนที่ใช้ในการสื่อการกับ Client ทั้งหมด ไม่ว่าจะเป็นการรับ หรือส่งข้อมูล โดยการรับข้อมูลจะมีการกรองข้อมูลที่ได้รับก่อนนำไปประมวล และมีการส่งข้อมูลไปยังไคลแอนท์เมื่อมีเงื่อนไข เช่น สั่งให้แอพพลิเคชั่นอัพเดท UI บอกสถานะการเชื่อต่อ ตอบโต้เพื่อยืนยันการส่งข้อมูล เป็นต้น             % แทรกโคด

		    [[subsubsection: System Controller]]
	   	    	เป็นตัวกลางระหว่าง Socket Controller กับ Hardware Controller ใช้ในการเก็บ และอัพเดทค่าต่างๆเช่น ความเร็ว มุมเลี้ยว เกียร์ปัจจุบัน โหมดปัจจุบันเป็นต้น ซึ่งค่าต่างๆจะได้มาจากการการประมวลผลข้อมูลที่รับมาจาก Socket Controller  
	            % แทรกโคด

			[[subsubsection: Hardware Controller]]
	        	ใช้ในการออกคำสั่งควบคุมอุปกรณ์ต่างๆซึ่งใช้ค่าดิจิตอลเพื่อสั่งงาน ได้แก่ มอเตอร์กระแสตรง เซอร์โวมอเตอร์ทั้ง 2 ตัว
	            % แทรกโคด

            
		[[subsubsection: ส่วนของฮาร์ดแวร์]]        
			ในส่วนของโครงรถจำลอง ได้ทำการดัดแปลงโครงรถเดิม โดยทำการออกแบบวงจรและอุปกรณ์ต่างๆเพื่อนำไปใส่ในโครงรถ จากนั้นจึงนำชิ้นส่วนบางอย่างอย่างที่ไม่จำเป็นออกไป เพื่อเพิ่มพื้นที่ ที่มีอยู่อย่างจำกัดให้เพียงพอต่ออุปกรณ์และวงจรที่นำไปวางใหม่
			[[subsection: โครงสร้างของของรถจำลอง]]
				ส่วนประกอบของรถจำลองมีดังนี้

				[[list]]
					#โครงรถประกอบด้วย
					[[ulist]]
						# กระจกข้าง ซ้ายขวา
						# กระจกมองหลัง
						# ล้อรถขนาด *** 4 ล้อ

					[[end]]
					# บอร์ด LinkiIt smart 7688 duo 
					# วงจรจ่ายไฟ
					# แบตเตอรี่ สำหรับจ่ายไฟให้ส่วนควบคุม
					# แบตเตอรี่ สำหรับจ่ายไฟให้ส่วนควบคุม
					# เซอร์โวมอเตอร์
					# มอเตอร์กระแสตรง
					# กล้องสำหรับถ่ายภาพเคลื่นไหว
			 
				[[end]]

%%%%%%%%%%%%%%%%%%%%  END 2.2  %%%%%%%%%%%%%%%

	[[subsection: การพัฒนาแอนดรอยแอพพลิเคชั่น ]]

		[[subsubsection: การออกแบบ UX/UI]]
	        การออกแบบ UI ของ Andriod Application เราได้จำลองสวนควบคุมรถที่สำคัญมาเป็นส่วน	ประกอบต่างๆบนหน้าจออุปกรณ์สมาร์ตโฟน ซึ่งได้แก่ เบรคและคันเร่ง พวงมาลัย และเกียร์
	        และเพิ่มส่วนประกอบอื่นๆที่สำคัญเข้าไปคือ ส่วนของการเปลี่ยนโหมดควบคุม และแสดงผลภาพเคลื่อนไหวจากกล้อง

	        [[list]]
	        # Home Activity
	         	[[image(home_ui):app/home.jpg|width=0.5|caption=ภาพหน้าแรกของ Application]]

	     	# Control Activity
	 			[[list]]
	        	# ช่องสำหรับกรอก IP มีค่าเริ่มต้นคือ 192.168.100.1
				# ปุ่มสำหรับเชื่อมต่อกับรถ(เซิฟเวอร์)
				# หากเชื่อมต่อไม่สำเร็จ จะปรากฏข้อความแจ้งเตือน
				[[end]]

				[[image(phone_control_ui):app/ph_con.jpg|width=0.5|caption=ภาพหน้าควบคุมรถโดย โทรศัพท์]]

				[[list]]
					# ชุดปุ่มสำหรับกดเปลี่ยนเกียร์
					# ปุ่มสไลด์สำหรับหมุนกล้อง
					# ปุ่มสำหรับสไลด์ ใช้ในควบคุมความเร็ว ซึ่งแยกเป็นเบรคและคันเร่ง
					# สวิทต์สำหรับเปลี่ยนโหมด
				[[end]]

				[[image(simset_control_ui):app/sim_con.jpg|width=0.5|caption=ภาพหน้าควบคุมรถโดย ชุดควบคุม]]
				จาก [[ref(simset_control_ui) ]] จะมีบางส่วนของการแสดงผลหายไป เหลือเพียงชุดปุ่มเพื่อแสดงสถานะของเกียร์ และสวิตท์สำหรับเปลี่ยนโหมดควบคุม


	        [[end]]

		[[subsubsection: การออกแบบและพัฒนาแอพพลิเคชั่นแอนดรอย ]]
			ในส่วนของซอฟต์แวร์พัฒนาด้วยภาษาจาวา และมีไลบารี่ของ Android SDK ซึ่งตัวภาษาของจาวาเองนั้นเป็นภาษาเชิงวัตถุ( Object-oriented programming ) โดยธรรมชาติอยู่แล้ว จึงทำให้การพัฒนาอยู่บนฐานของ Object-oriented สามารถแยกออกเป็นโมดูลต่างๆได้ดังนี้

			[[subsubsection: User Interface(UI) ]]

				User Interface(UI) เป็นส่วนที่ผู้ใช้ควบคุมรถ และแสดงผลให้ผู้ใช้ผ่านทางหน้าจอ ซึ่งการแสดงผลสามารถดูได้จาก [[ ref( การออกแบบ UX/UI) ]]
				ในส่วนของการรับค่าเพื่อนำไปประมวลผลนั้นมี 2 วิธี ได้แก่ การรับค่าจากส่วนตอบสนอง(UI)ผ่านหน้าจอสัมผัส และการอ่านค่าจากเซนเซอร์

				[[image(app_stucture_ui_ui_control):UI control.png|width=0.5|caption=UI Control ]]

			[[subsubsection: ส่วนของการทำงานเบื้องหลัง]]
			คือส่วนที่ทำงานเบื้องหลังโดยผู้ใช้จะไม่สามารถรับรู้การทำงานได้ ยกเว้นมีการอัพเดท UI โดยจะแบ่งเป็น 2 ส่วน
				
			[[list]]
				# การส่งและรับข้อมูล(Socket Programing) ใช้หลักการจาก [[ ref(subsection: การติดต่อสื่อสาร) ]] เพื่อรับและส่งข้อมูลโดยส่วนนี้จะต้องทำการสร้าง เธรด(Thread) ใหม่เพราะการพัฒนาแอพพลิเคชั่นบนแอนดรอยจไม่สามารถเชื่อมต่อผ่าน Socket บนเธรดหลัก ซึ่งเธรดที่ต้องสร้างใหม่มีทั้งหมด 2 เธรดซึ่งสร้างมาเพื่อรองกับการติดต่อ 2 ช่องทาง คือพอร์ท 7769 และ 7789 

				# ส่วนประมวลผลข้อมูล(Processing) ทุกๆครั้งที่การ รับ-ส่งข้อมูล ข้อมูลนั้นจะมีการตรวจสอบและถูกสั่งงานผ่านส่วนนี้

				# ส่วนของการเก็บค่าคงที่(Constant) เป็น Static class เพื่อที่จะเก็บตัวแปรและนำไปใช้ในคลาสอื่นๆได้อย่างสะดวก
			[[end]]

	[[ subsection: การพัฒนาชุดขับจำลอง ]]

		[[subsubsection: ส่วนของซฮฟต์แวร์]]


		[[subsubsection: ส่วนของฮาร์ดแวร์]]


%%%%%%%%%%%%%%%%%%%%  END 2  %%%%%%%%%%%%%%%
[[ section: ภาพรวมของโครงงาน]]

