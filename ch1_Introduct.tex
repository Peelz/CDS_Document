[[section: ความเป็นมา (หลักการและเหตุผล)]] 
% ให้เขียนในลักษณะ IPSA (Ideal -> Problem -> Solution -> Action)

    กิจกรรมขับรถเป็นกิจกรรมที่มีความน่าสนใจและมีความสนุกอยุ่ในตัว แต่การขับรถในชีวิตจริงนั้น ต้องใช้ทักษะและความสามารถ ความไม่ชำนาญของผู้ขับขี่  อาจเพิ่มโอกาสในการเกิดอุบัติเหตุ ซึ่งเป็นอันตรายต่อชีวิตและทรัพย์สิน ผู้ที่ต้องการขับรถนั้นอาจมีข้อจำกัดบางประการเช่นผู้ที่ไม่มีรถยนต์ ผู้ที่มีพื้นที่ที่จำกัดทำให้ไม่สามารถฝึกหัดขับขี่ได้ 
    จากปัญหาดังกล่าว หากนำเทคโนโลยีเข้าใช้เป็นอีกหนึ่งทางเลือกในการฝึกขับรถ ผู้พัฒนาจึงคิดค้นโครงงานนี้ขึ้นมา คือ ระบบควบคุมรถจำลองผ่านแอปพลิเคชันแอนดรอย (Car Driving Simulator System by Android Application) เพื่ออำนวยความสะดวก ให้กับผู้ที่ต้องการขับรถ ฝึกขับรถจำลองทดแทน 


[[section: วัตถุประสงค์]]

    โครงงานนี้จัดทำขึ้นมาเพื่อช่วยให้ผู้ที่ต้องการขับรถยนต์แต่มีข้อจำกัด  เช่น มีพื้นที่จำกัด  สามารถจะซ้อมขับรถจำลองได้ ในพื้นที่ของตนได้ ผู้ใช้สามารถฝึกทักษะการขับรถ และสามารถนำไปประยุกต์ใช้ในรถจริงทำให้ช่วยลดโอกาสเกิดอุบัติเหตุสำหรับผู้ที่ยังไม่มีความชำนาญในการขับรถยนต์  อีกทั้งผู้ที่ชื่นชอบในการขับรถ สามารถเพลิดเพลินกับการได้ขับรถจำลอง

[[section: ขอบเขตของโครงงาน]] 
    เป้าหมายของโครงการคือ ผู้ใช้สามารถใช้เสริมทักษะการขับรถได้ โดยทำโปรแกรมและระบบควบคุมรถจำลองสองระบบ ประกอบด้วย
    [[list]]
        # พัฒนาโครงการโดยใช้รถบังคับมอเตอร์ไฟฟ้ากระแสตรงขนาดเล็กเท่านั้น อัตราส่วนเทียบรถจริง 1:10  ขนาด กว้าง 20 เซนติเมตร  ยาว 46 เซนติเมตร สูง 15 เซนติเมตร น้ำหนัก 800 กรัม
        # ระยะการควบคุมไม่เกินรัศมีของสัญญาณไร้สาย ประมาณ 50 เมตร ขึ้นอยู่กับอุปกรณ์ android และ ระยะส่ง ของ microcontroller (Linklt Smart 7688 Duo) รองรับการสื่อสาร Wi-Fi 802.11 b/g/n
        # พัฒนาโปรแกรม Application บนระบบปฏิบัติการ Android  เท่านั้น
        # อุปกรณ์ระบบปฎิบัติการ android ที่ติดตั้ง Application ที่ใช้ควบคุมรถจำลองต้องมีเซนเซอร์ gyro และการบังคับเลี้ยวใช้มุม ซ้าย และ ขวา ด้านละ 90 องศาเท่านั้น โดยมีการคำนวนอัตราส่วนมุมเลี้ยวให้สอดคล้องกับรถจริง
        # การควบคุมรถจำลอง ทั้งสองระบบ ประกอบด้วย คันเร่ง เบรค พวงมาลัย และเกียร์แบบ automatic
        # ทำเฉพาะส่วนควบคุมรถหลัก คือ การใช้เกียร์แบบ automatic, การเบรค, การเร่ง โปรแกรมสั่งงานผ่านมอเตอร์ไฟฟ้า  การเลี้ยวโปรแกรมสั่งงานผ่าน Servo
        # ภาพได้จากกล้องหนึ่งตัวมุมมองเดิม
    [[end]]

% From Implementation %
    1. โปรแกรมทำงานรับส่ง ไม่เกินรัศมีของสัญญาณไร้สาย ประมาณ 50 เมตร ขึ้นอยู่กับอุปกรณ์แอนดรอย และ ระยะส่ง Linklt Smart 7688 Wi-Fi 802.11 b/g/n
    2. อุปกรณ์ที่ติดตั้งโปรแกรมต้องใช้ระบบปฎิบัติการ android และต้องมีเซนเซอร์ gyro 
    3. อาจมี delay เนื่องจากระยะทางและความเข้มของสัญญานรับส่ง
    4. โปรแกรมรับภาพที่ได้จากกล้องหนึ่งตัวมุมมองเดิม

[[section: กลุ่มผู้ใช้งาน]] 
    [[list]]
        # บุคคลทั่วไป
    [[end]]

[[section: ประโยชน์ที่ได้รับ]] 
    [[list]]
        # ผู้ใช้สามารถใช้ระบบของโครงงานนี้ในการฝึกขับรถ แทนการขับรถจริงได้
        # ความเพลิดเพลินและความสนุก ขณะขับรถ
    [[end]]

[[section: แผนการดำเนินงาน]] 
    % แจกแจงขั้นตอนการดำเนินงานที่ได้วางแผนไว้ ควรเขียนในรูปแบบรายการ ตามลำดับ
    [[ulist]]
        # ออกแบบโครงสร้างการทำงานอย่างคร่าวๆ
        # ศึกษาการทำงานของภาษา Java(Android) และ Python
        # ออกแบบ Protocal การสื่อสารระหว่าง ตัวควบคุมและรถ
        # ออกแบบการทำงานของแอพพลิเคชชั่นบนมือถือ และดำเนิการสร้างเป็นแอพพลิเคชั่นจริงๆ
        # ออกแบบและสร้างรถจำลอง
        # ออกแบบและสร้างชุดขับจำลอง
        # ทดสอบระบบโดยแยกเป็นส่วนย่อยๆก่อนนำมารวมกันแล้วทดสอบจริง
        # วิเคราะห์และปรับปรุงข้อผิดพลาด
        # สรุปผลการทดลองการใช้งานจริง
    [[end]]


