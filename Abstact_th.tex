บทคัดย่อ

โครงงานนี้มีวัตถุประสงค์หลักในการจำลองการขับรถโดยใช้รถจำลองที่มีรูปแบบการเคลื่อนที่เสมือนกับรถจริง และควบคุมด้วยระบบจำลองการบังคับเสมือนจริง โดยที่ระบบประกอบไปด้วย 3 ส่วนหลักๆคือ 
ส่วนของรถจำลอง พัฒนาโดยใช้ บอร์ด LinkItSmart7688 Duo เป็นอุปกรณ์หลัก โดยใช้ภาษา Python ในการพัฒนาบนระบบปฏิบัติการ Linux(openWRT)เป็นเซิฟเวอร์ในการรับคำสั่งจากสัญญาณไร้สายที่บอร์ดเป็น Access point เพื่อควบคุม มอเตอร์ด้วย Pulse wide modulation และส่งภาพ MJPEG ไปยังโทรศัพท์สมาร์ทโฟนผ่าน HTTP
ส่วนของแอพพลิเคชั่นบนสมาร์ทโฟน พัฒนาบนโทรศัพท์ที่ใช้ระบบปฏิบัติการ Android ซึ่งใช้ภาษา Java ในการพัฒนา โดยในส่วนแอพพลิเคชั่นจะรับสัญญาณภาพจากรถจำลองและแสดงผลทางหน้าจอ และเป็นตัวควบคุมรถอีกด้วย 
ชุดควบคุมรถสเมือนจริง เป็นชุดออกคำสั่งควบคุมรถโดยสร้างให้เหมือนกับการขับรถจริงๆ โดยมีอุปกรณ์ในการรับคำสั่งจากพวงมาลัย คันเร่ง เบรค เกียร์อัตโนมัติ และสั่งงานผ่านสัญญาณไร้สายเพื่อควบคุมรถจำลอง

