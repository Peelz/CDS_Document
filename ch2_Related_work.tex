[[section: Python]] 
	Python เป็นภาษาโปรแกรมคอมพิวเตอร์ระดับสูง สำหรับใช้งานทั่วไป
	โดยจุดเด่นของ Python คือการที่เป็นภาษาสคริปต์ (Scripting Language)
	ทำให้ใช้เวลาในการเขียนและคอมไพล์ไม่มาก

[[section: pyFirmata]]
    เป็น Libery สำหรับภาษา Python ใช้สำหรับติดต่อกับบอร์ดตระกูล Arduino zjko

[[section: Java ]]
    Java หรือ Java programming language คือภาษาโปรแกรมเชิงวัตถุ พัฒนาโดย เจมส์ กอสลิง และวิศวกรคนอื่นๆ ที่บริษัท ซัน ไมโครซิสเต็มส์ ภาษานี้มีจุดประสงค์เพื่อใช้แทนภาษาซีพลัสพลัส C++ โดยรูปแบบที่เพิ่มเติมขึ้นคล้ายกับภาษาอ็อบเจกต์ทีฟซี (Objective-C) แต่เดิมภาษานี้เรียกว่า ภาษาโอ๊ก (Oak) ซึ่งตั้งชื่อตามต้นโอ๊กใกล้ที่ทำงานของ เจมส์ กอสลิง แล้วภายหลังจึงเปลี่ยนไปใช้ชื่อ "จาวา" ซึ่งเป็นชื่อกาแฟแทน จุดเด่นของภาษา Java อยู่ที่ผู้เขียนโปรแกรมสามารถใช้หลักการของ Object-Oriented Programming มาพัฒนาโปรแกรมของตนด้วย Java ได้ 
    ภาษา Java เป็นภาษาสำหรับเขียนโปรแกรมที่สนับสนุนการเขียนโปรแกรมเชิงวัตถุ ( OOP : Object-Oriented Programming) โปรแกรมที่เขียนขึ้นถูกสร้างภายในคลาส ดังนั้นคลาสคือที่เก็บเมทอด (Method) หรือพฤติกรรม (Behavior) ซึ่งมีสถานะ (State) และรูปพรรณ (Identity) ประจำพฤติกรรม (Behavior) 
    ที่มา http://www.mindphp.com/%E0%B8%84%E0%B8%B9%E0%B9%88%E0%B8%A1%E0%B8%B7%E0%B8%AD/73-%E0%B8%84%E0%B8%B7%E0%B8%AD%E0%B8%AD%E0%B8%B0%E0%B9%84%E0%B8%A3/2185-java-%E0%B8%84%E0%B8%B7%E0%B8%AD%E0%B8%AD%E0%B8%B0%E0%B9%84%E0%B8%A3.html

[[section: Android SDK]]
    ย่อมาจาก Android Software Development Kit พัฒนาและแจกจายฟรีไม่มีค่าใช้จ่ายโดย Google สำหรับเป็นเครื่องมือสำหรับพัฒนาแอพพลิเคชั่นบน Android ผู้พัฒนาสามารถพัฒนาแอพพลิเคชั่นได้อย่างรวดเร็วและมี Emulator สำหรับจำลองอุปกรณ์ Android บนเครื่องคอมพิวเตอร์ที่ใช้พัฒนาได้อีกด้วย

[[section: MJPEG Viewer]]
    เป็น Libery สำเร็จรูปในการโหลดข้อมูลจาก HTTP Sreaming แล้วตรวจจับข้อมูลภาพสกุล JPEG ที่ละภาพจากนั้นนำมาแสดงผลบนหน้าจอ


[[section: MJPEG และ Magic number]
    ย่อมาจาก Motion JPEG เป็นรูปแบบภาพเคลื่อนไหวที่่เกิดจากาการนำภาพ JPEG มาเรียงต่อกันที่ละเฟรม

    เราสามารถแยก MJPEG เป็น JPEG ได้โดยการอ่าน Buffer ซึ่งไฟล์ JPEG จะมีชุดรหัสเริ่มต้นคือ _oxFF oxD8_ และ จบด้วย _oxFF oxD9_

[[section: LinkIt Smart 7688 Duo (Linux Embedded)]]

    [[image(7688-duo_spec):ch2_7688_duo_all_spec.jpg|width=0.5|caption=LinkIt Smart 7688 Duo]]
    จากภาพ [[ref(7688-duo_spec)]], แสดงรายละเอียดของบอร์ด LinkIt Smart 7688 Duo.

    Linux Embedded คืออุปกรณ์สมองกลฝังตัวมีระบบปฏิบัติ Linux เป็นซอฟแวร์ในการจัดการทรัพยากร 
    โดยที่ LinkIt Smart 7688 Duo เป็นบอร์ด Embedded ระบบปฏิบัติการ Linux OpenWrt  เหมาะสำหรับอุปกรณ์ที่เน้นการเชื่อมต่อเครือข่ายคอมพิวเตอร์ 
    หรือประยุกต์ใช้งานกับระบบที่เชื่อมต่อกับเครือข่ายทั้งมีสายและไร้สาย ผลิตโดยความร่วมมือของ MediaTek และ SeeedStudio ใช้ชิพ MT7688 ของ MediaTek 
    สถาปัตยกรรม MIPS ความเร็ว 580 MHz มีหน่วยความจำ DDR2 ขนาด 128MB มี Flash Memory ขนาด 32 MB พร้อม Wi-Fiมาตรฐาน 802.11bgn พร้อมขาสัญญาณต่างๆ 
    ให้ใช้งาน ได้แก่ GPIO I2C SPI UART PWM Ethernet USB Host และ Micro SD Card Slot


[[section: H-bridge Circuit]] 
    [[image(h-bridge):ch_h-bridge.gif|width=0.5|caption=วงจร H-Bridge]]
    % อธิบายวงจร

[[subsection: Camera]]
    [[image(Car_camera):ch2_web_cam.png|width=0.5|caption=กล้องบนโมเดลรถ ]]
    คุณสมบัติ
    [[ulist]]
        # ความละเอียด: 2.0M พิกเซล 12.0M พิกเซล จากโปรแกรมคอมพิวเตอร์
        # อัตราต่อเฟรม:1920*1080 ต่อ30fps
        # อินเตอร์เฟซ : USB 2.0
        # อัตราเสียง: 48dB
        # คุณภาพเลนส์: สูง
        # ระยะโฟกัส: 30mm-อินฟินิตี้
        # ฟังก์ชั่นปรับความคมชัดอัตโนมัติในตัว Auto Focus 
    [[end]]

[[section: Wi-Fi  ( Wireless-Fidelity)  ]]
    เป็นเทคโนโลยีเครือข่ายไร้สาย  ภายใต้เทคโนโลยีการสื่อสาร มาตรฐาน IEEE  802.11  โดยเป็นมาตรฐานที่ถูกอนุมัติให้ ใช้จาก IEEE (The Institute of Electrical and Electronics Engineers) 
    เพื่อให้อุปกรณ์คอมพิวเตอร์สามารถสื่อสารกันได้บนมาตรฐานการทำงานแบบเดียวกัน  สำหรับเทคโนโลยีเครือข่ายไร้สายนี้จะใช้คลื่นความถี่ RF 
    และคลื่นความถี่อินฟาเรตในการรับส่งข้อมูลคลื่นความถี่วิทยุของเครือข่ายไร้สายจึงสามารถทะลุทะลวงกำแพงหรือสิ่งกีดขวางได้ทำให้การใช้งานบนเครือข่ายไร้สายมีความคล่องตัวและสะดวกสบายมากขึ้น 
    โดยสามารถเชื่อมต่อเข้าสู่เครือข่ายได้ทุกที่ที่มีคลื่นสัญญาณ  ข้อมูลจะถูกรับส่งผ่านคลื่นวิทยุความถี่   2.4 GHz  ด้วยความเร็ว  11 Mbps  ระยะห่างประมาณ  300 ฟุต นำมาใช้ในการรับส่งข้อมูลระหว่างตัวรถและชุดบังคับ
 
[[section: HTTP Live Streaming]]
    การสตรีมมิ่งวิดีโอ คือการแสดงผลภาพเคลื่อนไหวจากกล้องหรือไฟล์ไปยังลูกข่าย(Client) โดยที่ผู้ชมจะได้รับการแสดงผลแบบ Real timne ซึ่งการสตรีมมิ่งผ่าน HTTP ถือว่าเป็นอีกหนึ่งช่องทางที่ได้รับความนิยม 
    เนื่องจากความสะดวกในการพัฒนาและมีอุปกรณ์รองรับได้เป็นจำนวนมาก

[[section: Servo Motor]]
    [[image(servo):ch2_servo.png|width=0.5|caption=Servo Motor]]
    คือ Motor ที่สามารถสั่งงานให้หมุนไปยังตำแหน่งองศาที่ต้องการถูกต้อง โดยใช้การควบคุมแบบป้อนกลับ (Feedback Control)  Feedback Control 

 [[section: ]]   
    คือ ระบบควบคุมที่มีการวัดค่าเอาต์พุตของระบบนำมาเปรียบเทียบกับค่าอินพุตเพื่อควบคุมและปรับแต่งให้ค่าเอาต์พุตของระบบให้มีค่า เท่ากับ หรือ ใกล้เคียงกับค่าอินพุต Servo motor 
    มีสายเชื่อมต่อเพื่อ จ่ายไฟฟ้า และ ควบคุม จะประกอบด้วยสายไฟ 3 เส้น คือ ไฟเลี้ยง (4.8-6V) กราวด์ และสายส่งสัญญาณพัลซ์ควบคุม (3-5V) 
    หลักการทำงานของ RC Servo Motor เมื่อจ่ายสัญญาณพัลซ์เข้ามายัง RC Servo Motor ส่วนวงจรควบคุม (Electronic Control System) ภายใน Servo 
    จะทำการอ่านและประมวลผลค่าความกว้างของสัญญาณพัลซ์ PWM (Pulse Width Modulation) ที่ส่งเข้ามาเพื่อแปลค่าเป็นตำแหน่งองศาที่ต้องการให้ Motor 
    หมุนเคลื่อนที่ไปยังตำแหน่งนั้น แล้วส่งคำสั่งไปทำการควบคุมให้ Motor หมุนไปยังตำแหน่งที่ต้องการ โดยมี Position Sensor เป็นตัวเซ็นเซอร์คอยวัดค่ามุมที่ Motor กำลังหมุน 
    เป็น Feedback กลับมาให้วงจรควบคุมเปรียบเทียบกับค่าอินพุตเพื่อควบคุมให้ได้ตำแหน่งที่ต้องการอย่างถูกต้องแม่นยำ
    ยกตัวอย่างเช่นหากกำหนดความกว้างของสัญญาณพัลซ์ไว้ที่ 1 ms ตัว Servo Motor จะหมุนไปทางด้ายซ้ายจนสุด ในทางกลับกันหากกำหนดความกว้างของสัญญาณพัลซ์ไว้ที่ 2 ms 
    ตัว Servo Motor จะหมุนไปยังตำแหน่งขวาสุด แต่หากกำหนดความกว้างของสัญญาณพัลซ์ไว้ที่ 1.5 ms ตัว Servo Motor ก็จะหมุนมาอยู่ที่ตำแหน่งตรงกลางพอดี ดังนั้นสามารถกำหนดองศาการหมุนของ Servo Motor ได้โดยการเทียบค่า 
    เช่น Servo Motor สามารถหมุนได้ 180 องศา โดยที่ 0 องศาใช้ความกว้างพัลซ์เท่ากับ 1000 us ที่ 180 องศาความกว้างพัลซ์เท่ากับ 2000 us เพราะฉะนั้นค่าที่เปลี่ยนไป 1 องศาจะใช้ความกว้างพัลซ์ต่างกัน (2000-1000)/180 เท่ากับ 5.55 us 
    จากการหาค่าความกว้างพัลซ์ที่มุม 1 องศาข้างต้น หากต้องกำหนดให้ RC Servo Motor หมุนไปที่มุม 45 องศาจะหาค่าพัลซ์ที่ต้องการได้จาก 5.55 x 45 เท่ากับ 249.75 us 
    แต่ที่มุม 0 องศาเราเริ่มที่ความกว้างพัลซ์ 1ms หรือ 1000 us เพราะฉะนั้นความกว้างพัลซ์ที่ใช้กำหนดให้ RC Servo Motor หมุนไปที่ 45 องศา คือ 1000 + 249.75 เท่ากับประมาณ 1250 us

[[section: ]]