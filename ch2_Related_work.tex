[[section: Python]] 
	Python เป็นภาษาโปรแกรมคอมพิวเตอร์ระดับสูง สำหรับใช้งานทั่วไป
	โดยจุดเด่นของ Python คือการที่เป็นภาษาสคริปต์ (Scripting Language)
	ทำให้ใช้เวลาในการเขียนและคอมไพล์ไม่มาก

[[section: Android Application]]
    แอนดรอย คือชื่อระบบปฏิบัติที่ถูกออกแบบมาสำหรับอุปกรณ์ที่ใช้จอสัมผัสเช่น สมาร์ตโฟน และแท็บเล็ตคอมพิวเตอร์ ผู้พัฒนาสามารถพัฒนาโปรแกรมประยุกต์บนอุปกรณ์แอนดรอยได้หลากหลายแบบ 
    แต่ในโครงงานนี้จะเลือกใช้ภาษา Java ซึ่งเป็นภาษาหลักในการพัฒนาโปรแกรมประยุกต์บนระบบปฏิบัติการแอนดรอย


[[section: LinkIt Smart 7688 Duo (Linux Embedded)]]
    Linux Embedded คืออุปกรณ์สมองกลฝังตัวมีระบบปฏิบัติ Linux เป็นซอฟแวร์ในการจัดการทรัพยากร 
    โดยที่ LinkIt Smart 7688 Duo เป็นบอร์ด Embedded ระบบปฏิบัติการ Linux OpenWrt  เหมาะสำหรับอุปกรณ์ที่เน้นการเชื่อมต่อเครือข่ายคอมพิวเตอร์ 
    หรือประยุกต์ใช้งานกับระบบที่เชื่อมต่อกับเครือข่ายทั้งมีสายและไร้สาย ผลิตโดยความร่วมมือของ MediaTek และ SeeedStudio ใช้ชิพ MT7688 ของ MediaTek 
    สถาปัตยกรรม MIPS ความเร็ว 580 MHz มีหน่วยความจำ DDR2 ขนาด 128MB มี Flash Memory ขนาด 32 MB พร้อม Wi-Fiมาตรฐาน 802.11bgn พร้อมขาสัญญาณต่างๆ 
    ให้ใช้งาน ได้แก่ GPIO I2C SPI UART PWM Ethernet USB Host และ Micro SD Card Slot

[[section: Wi-Fi  ( Wireless-Fidelity)  ]]
    เป็นเทคโนโลยีเครือข่ายไร้สาย  ภายใต้เทคโนโลยีการสื่อสาร มาตรฐาน IEEE  802.11  โดยเป็นมาตรฐานที่ถูกอนุมัติให้ ใช้จาก IEEE (The Institute of Electrical and Electronics Engineers) 
    เพื่อให้อุปกรณ์คอมพิวเตอร์สามารถสื่อสารกันได้บนมาตรฐานการทำงานแบบเดียวกัน  สำหรับเทคโนโลยีเครือข่ายไร้สายนี้จะใช้คลื่นความถี่ RF 
    และคลื่นความถี่อินฟาเรตในการรับส่งข้อมูลคลื่นความถี่วิทยุของเครือข่ายไร้สายจึงสามารถทะลุทะลวงกำแพงหรือสิ่งกีดขวางได้ทำให้การใช้งานบนเครือข่ายไร้สายมีความคล่องตัวและสะดวกสบายมากขึ้น 
    โดยสามารถเชื่อมต่อเข้าสู่เครือข่ายได้ทุกที่ที่มีคลื่นสัญญาณ  ข้อมูลจะถูกรับส่งผ่านคลื่นวิทยุความถี่   2.4 GHz  ด้วยความเร็ว  11 Mbps  ระยะห่างประมาณ  300 ฟุต นำมาใช้ในการรับส่งข้อมูลระหว่างตัวรถและชุดบังคับ
 
[[section: HTTP Live Streaming]]
    การสตรีมมิ่งวิดีโอ คือการแสดงผลภาพเคลื่อนไหวจากกล้องหรือไฟล์ไปยังลูกข่าย(Client) โดยที่ผู้ชมจะได้รับการแสดงผลแบบ Real timne ซึ่งการสตรีมมิ่งผ่าน HTTP ถือว่าเป็นอีกหนึ่งช่องทางที่ได้รับความนิยม 
    เนื่องจากความสะดวกในการพัฒนาและมีอุปกรณ์รองรับได้เป็นจำนวนมาก

[[section: ]]
    คือ Motor ที่สามารถสั่งงานให้หมุนไปยังตำแหน่งองศาที่ต้องการถูกต้อง โดยใช้การควบคุมแบบป้อนกลับ (Feedback Control)  Feedback Control 
    คือ ระบบควบคุมที่มีการวัดค่าเอาต์พุตของระบบนำมาเปรียบเทียบกับค่าอินพุตเพื่อควบคุมและปรับแต่งให้ค่าเอาต์พุตของระบบให้มีค่า เท่ากับ หรือ ใกล้เคียงกับค่าอินพุต Servo motor 
    มีสายเชื่อมต่อเพื่อ จ่ายไฟฟ้า และ ควบคุม จะประกอบด้วยสายไฟ 3 เส้น คือ ไฟเลี้ยง (4.8-6V) กราวด์ และสายส่งสัญญาณพัลซ์ควบคุม (3-5V) 
    หลักการทำงานของ RC Servo Motor เมื่อจ่ายสัญญาณพัลซ์เข้ามายัง RC Servo Motor ส่วนวงจรควบคุม (Electronic Control System) ภายใน Servo 
    จะทำการอ่านและประมวลผลค่าความกว้างของสัญญาณพัลซ์ PWM (Pulse Width Modulation) ที่ส่งเข้ามาเพื่อแปลค่าเป็นตำแหน่งองศาที่ต้องการให้ Motor 
    หมุนเคลื่อนที่ไปยังตำแหน่งนั้น แล้วส่งคำสั่งไปทำการควบคุมให้ Motor หมุนไปยังตำแหน่งที่ต้องการ โดยมี Position Sensor เป็นตัวเซ็นเซอร์คอยวัดค่ามุมที่ Motor กำลังหมุน 
    เป็น Feedback กลับมาให้วงจรควบคุมเปรียบเทียบกับค่าอินพุตเพื่อควบคุมให้ได้ตำแหน่งที่ต้องการอย่างถูกต้องแม่นยำ
    ยกตัวอย่างเช่นหากกำหนดความกว้างของสัญญาณพัลซ์ไว้ที่ 1 ms ตัว Servo Motor จะหมุนไปทางด้ายซ้ายจนสุด ในทางกลับกันหากกำหนดความกว้างของสัญญาณพัลซ์ไว้ที่ 2 ms 
    ตัว Servo Motor จะหมุนไปยังตำแหน่งขวาสุด แต่หากกำหนดความกว้างของสัญญาณพัลซ์ไว้ที่ 1.5 ms ตัว Servo Motor ก็จะหมุนมาอยู่ที่ตำแหน่งตรงกลางพอดี ดังนั้นสามารถกำหนดองศาการหมุนของ Servo Motor ได้โดยการเทียบค่า 
    เช่น Servo Motor สามารถหมุนได้ 180 องศา โดยที่ 0 องศาใช้ความกว้างพัลซ์เท่ากับ 1000 us ที่ 180 องศาความกว้างพัลซ์เท่ากับ 2000 us เพราะฉะนั้นค่าที่เปลี่ยนไป 1 องศาจะใช้ความกว้างพัลซ์ต่างกัน (2000-1000)/180 เท่ากับ 5.55 us 
    จากการหาค่าความกว้างพัลซ์ที่มุม 1 องศาข้างต้น หากต้องกำหนดให้ RC Servo Motor หมุนไปที่มุม 45 องศาจะหาค่าพัลซ์ที่ต้องการได้จาก 5.55 x 45 เท่ากับ 249.75 us 
    แต่ที่มุม 0 องศาเราเริ่มที่ความกว้างพัลซ์ 1ms หรือ 1000 us เพราะฉะนั้นความกว้างพัลซ์ที่ใช้กำหนดให้ RC Servo Motor หมุนไปที่ 45 องศา คือ 1000 + 249.75 เท่ากับประมาณ 1250 us
