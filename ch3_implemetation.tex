
6.1 เนื้อเรื่องย่อ (Story Board) 
ภาพแสดงการสื่อสารทั้งสองระบบ

บนรถจำลองมีกล้องที่ต่อกับ Linkit Smart7688Duo ซึ่งเป็นอุปกรณ์ปล่อยสัญญาน wireless access point (Wi-Fi 802.11 b/g/n)  และเป็น controller ควบคุมการขับเคลื่อนและการเลี้ยว 
เมื่อมีอุปกรณ์ควบคุม เชื่อมต่อกับ Access point  โปรแกรมจะส่งข้อมูล เข้ามาที่ controller โดย controller  จะใช้สัญญาน Pulse Width Modulation ในการควบคุม DC motor ส่งผลต่อความเร็วรถจำลอง และ ควบคุมองศาของ servo ส่งผลต่อการเลี้ยวของรถรถจำลอง โดยอุปกรณ์ควบคุมนั้นแบ่งเป็นสองระบบคือ ระบบ Android และ ระบบชุดพวงมาลัยควบคุม 



การควบคุมจากโปรแกรมบนอุปกรณ์พกพาระบบปฎิบัติการ android 


ภาพแสดง user interface เบื้องต้น ประกอบด้วยเกียร์แบบ automatic เบรค คันเร่ง

Application ที่ติดตั้งบนอุปกรณ์พกพาระบบปฎิบัติการ android สามารถสั่งงานควบคุมคันเร่ง เบรค และเกียร์ automatic โดยใช้การสัมผัสหน้าจอ ควบคุมการเลี้ยวโดยใช้การเอียง โปรแกรมรับค่าจากเซนเซอร์gyro เพื่อคำนวนองศาเลี้ยว โปรแกรมจะคำนวนอัตราส่วนมุมเลี้ยว และความเร็วรถ  จากนั้นส่งข้อมูลการควบคุมไร้สายไปยัง Linkit smart7688duo   เมื่อ Linkit smart7688duo ได้รับคำสั่งโปรแกรม จะป้อนสัญญาน Pulse Width Modulation เพื่อควบคุมวงจรสั่งงาน DC motor ขับเคลื่อน และ servo เลี้ยวต่อไป 
ส่วนภาพจะถูกส่งจากกล้อง ไปยังอุปกรณ์พกพาระบบปฏิบัติการ android นำขึ้นจอภาพ android โดยใช้ ระบบ MJPG-streamer


โหมดการควบคุมจากโปรแกรมบนชุดพวงมาลัยควบคุม 

ภาพแสดงชุดพวงมาลัยควบคุม ประกอบด้วยพวงมาลัย คันเร่ง เบรค และเกียร์ ส่วนรับภาพเป็นแว่น VR

โปรแกรมที่ติดตั้งบนระบบชุดพวงมาลัยควบคุม สามารถสั่งงานควบคุมคันเร่ง เบรค และเกียร์ automatic โดยใช้กลไกพวงมาลัย เกียร์ คันเร่งและเบรคจริง มาประมวลผล โปรแกรมจะคำนวนอัตราส่วนมุมเลี้ยว และความเร็วรถ  จากนั้นส่งข้อมูลการควบคุมไร้สายไปยัง Linkit smart7688duo   เมื่อ Linkit smart7688duo ได้รับคำสั่งโปรแกรม จะป้อนสัญญาน Pulse Width Modulation เพื่อควบคุมวงจรสั่งงาน DC motor ขับเคลื่อน และ servo เลี้ยวต่อไป 
ส่วนภาพจะถูกส่งจากกล้อง ไปยังอุปกรณ์พกพาระบบปฏิบัติการ android นำขึ้นจอภาพ android แบบ แว่น Virtual reality โดยใช้ ระบบ MJPG-streamer 


6.2 เทคนิคหรือเทคโนโลยีที่ใช้ 
Wi-Fi  ( Wireless-Fidelity)  เป็นเทคโนโลยีเครือข่ายไร้สาย  ภายใต้เทคโนโลยีการสื่อสาร มาตรฐาน IEEE  802.11  โดยเป็นมาตรฐานที่ถูกอนุมัติให้ ใช้จาก IEEE (The Institute of Electrical and Electronics Engineers) เพื่อให้อุปกรณ์คอมพิวเตอร์สามารถสื่อสารกันได้บนมาตรฐานการทำงานแบบเดียวกัน  สำหรับเทคโนโลยีเครือข่ายไร้สายนี้จะใช้คลื่นความถี่ RF และคลื่นความถี่อินฟาเรตในการรับส่งข้อมูลคลื่นความถี่วิทยุของเครือข่ายไร้สายจึงสามารถทะลุทะลวงกำแพงหรือสิ่งกีดขวางได้ทำให้การใช้งานบนเครือข่ายไร้สายมีความคล่องตัวและสะดวกสบายมากขึ้น โดยสามารถเชื่อมต่อเข้าสู่เครือข่ายได้ทุกที่ที่มีคลื่นสัญญาณ  ข้อมูลจะถูกรับส่งผ่านคลื่นวิทยุความถี่   2.4 GHz  ด้วยความเร็ว  11 Mbps  ระยะห่างประมาณ  300 ฟุต นำมาใช้ในการรับส่งข้อมูลระหว่างตัวรถและชุดบังคับ
 
สตรีมมิ่งมีเดีย (Streaming Media) ถูกพัฒนาขึ้นมา โดยเปลี่ยนรูปแบบการดาวน์โหลดข้อมูลที่ต้องรอให้การโหลดข้อมูลเสร็จสิ้นก่อน จึงจะแสดงผล ให้แสดงผลลัพธ์ไปพร้อมๆกับการโหลดข้อมูล ใช้ในการรับส่งสัญญานภาพระหว่างรถจำลองกับชุดควบคุม
Servo Motor คือ Motor ที่สามารถสั่งงานให้หมุนไปยังตำแหน่งองศาที่ต้องการถูกต้อง โดยใช้การควบคุมแบบป้อนกลับ (Feedback Control)  Feedback Control คือ ระบบควบคุมที่มีการวัดค่าเอาต์พุตของระบบนำมาเปรียบเทียบกับค่าอินพุตเพื่อควบคุมและปรับแต่งให้ค่าเอาต์พุตของระบบให้มีค่า เท่ากับ หรือ ใกล้เคียงกับค่าอินพุต Servo motor มีสายเชื่อมต่อเพื่อ จ่ายไฟฟ้า และ ควบคุม จะประกอบด้วยสายไฟ 3 เส้น คือ ไฟเลี้ยง (4.8-6V) กราวด์ และสายส่งสัญญาณพัลซ์ควบคุม (3-5V) 
หลักการทำงานของ RC Servo Motor เมื่อจ่ายสัญญาณพัลซ์เข้ามายัง RC Servo Motor ส่วนวงจรควบคุม (Electronic Control System) ภายใน Servo จะทำการอ่านและประมวลผลค่าความกว้างของสัญญาณพัลซ์ PWM (Pulse Width Modulation) ที่ส่งเข้ามาเพื่อแปลค่าเป็นตำแหน่งองศาที่ต้องการให้ Motor หมุนเคลื่อนที่ไปยังตำแหน่งนั้น แล้วส่งคำสั่งไปทำการควบคุมให้ Motor หมุนไปยังตำแหน่งที่ต้องการ โดยมี Position Sensor เป็นตัวเซ็นเซอร์คอยวัดค่ามุมที่ Motor กำลังหมุน เป็น Feedback กลับมาให้วงจรควบคุมเปรียบเทียบกับค่าอินพุตเพื่อควบคุมให้ได้ตำแหน่งที่ต้องการอย่างถูกต้องแม่นยำ
ยกตัวอย่างเช่นหากกำหนดความกว้างของสัญญาณพัลซ์ไว้ที่ 1 ms ตัว Servo Motor จะหมุนไปทางด้ายซ้ายจนสุด ในทางกลับกันหากกำหนดความกว้างของสัญญาณพัลซ์ไว้ที่ 2 ms ตัว Servo Motor จะหมุนไปยังตำแหน่งขวาสุด แต่หากกำหนดความกว้างของสัญญาณพัลซ์ไว้ที่ 1.5 ms ตัว Servo Motor ก็จะหมุนมาอยู่ที่ตำแหน่งตรงกลางพอดี ดังนั้นสามารถกำหนดองศาการหมุนของ Servo Motor ได้โดยการเทียบค่า เช่น Servo Motor สามารถหมุนได้ 180 องศา โดยที่ 0 องศาใช้ความกว้างพัลซ์เท่ากับ 1000 us ที่ 180 องศาความกว้างพัลซ์เท่ากับ 2000 us เพราะฉะนั้นค่าที่เปลี่ยนไป 1 องศาจะใช้ความกว้างพัลซ์ต่างกัน (2000-1000)/180 เท่ากับ 5.55 us 
จากการหาค่าความกว้างพัลซ์ที่มุม 1 องศาข้างต้น หากต้องกำหนดให้ RC Servo Motor หมุนไปที่มุม 45 องศาจะหาค่าพัลซ์ที่ต้องการได้จาก 5.55 x 45 เท่ากับ 249.75 us แต่ที่มุม 0 องศาเราเริ่มที่ความกว้างพัลซ์ 1ms หรือ 1000 us เพราะฉะนั้นความกว้างพัลซ์ที่ใช้กำหนดให้ RC Servo Motor หมุนไปที่ 45 องศา คือ 1000 + 249.75 เท่ากับประมาณ 1250 us


Linkit Smart 7688Duo   เป็นบอร์ด Embedded ระบบปฏิบัติการ Linux OpenWrt  เหมาะสำหรับอุปกรณ์ที่เน้นการเชื่อมต่อเครือข่ายคอมพิวเตอร์ หรือประยุกต์ใช้งานกับระบบที่เชื่อมต่อกับเครือข่ายทั้งมีสายและไร้สาย ผลิตโดยความร่วมมือของ MediaTek และ SeeedStudio ใช้ชิพ MT7688 ของ MediaTek สถาปัตยกรรม MIPS ความเร็ว 580 MHz มีหน่วยความจำ DDR2 ขนาด 128MB มี Flash Memory ขนาด 32 MB พร้อม Wi-Fiมาตรฐาน 802.11bgn พร้อมขาสัญญาณต่างๆ ให้ใช้งาน ได้แก่ GPIO I2C SPI UART PWM Ethernet USB Host และ Micro SD Card Slot


6.3 เครื่องมือที่ใช้ในการพัฒนา
อุปกรณ์ด้านซอฟแวร์
	Arduino IDE
	Android studio
	Python  
	Java 
	อุปกรณ์ด้านฮาร์ดแวร์
	รถจำลอง
	ชุดพวงมาลัยควบคุม
	DC motor
	Servo motor
	Linkit Smart 7688Duo 
 
6.4 รายละเอียดโปรแกรมที่จะพัฒนา (Software Specification) 
Input 
	-ภาพจากกล้องบนรถจำลอง
	-หมวดการควบคุมจากโปรแกรมบนอุปกรณ์พกพาระบบปฎิบัติการ android 
	input เป็นระบบสัมผัส เร่ง เบรค เปลี่ยนเกียร์ และการเอียงเลี้ยว
	-หมวดการควบคุมจากโปรแกรมบนชุดพวงมาลัยควบคุม
	Input เป็นคันเร่ง เบรค เกียร์ และพวงมาลัย

Output 
	-ฝั่งรถจำลองเป็นการขับเคลื่อนและการเลี้ยว
	-ฝั่งรับภาพ แสดงบนหน้าจอ หรือ หน้าจอบนแว่น Virtual reality



        	        
Functional Specification
	-โปรแกรมแบ่งเป็นสองระบบคือควบคุมผ่านอุปกรณ์ระบบปฎิบัติการ android และควบคุมผ่านชุดพวงมาลัย
	-สามารถควบคุมความเร็วได้
	-สามารถควบคุมองศาการเลี้ยวได้
	-สามารถควบคุมเกียร์ระบบ automaticได้ ประกอบด้วย เกียร์ P, R, N, D
	-เห็นภาพจากกล้องที่ติดบนรถจำลอง

